\documentclass[]{article}
\usepackage{lmodern}
\usepackage{amssymb,amsmath}
\usepackage{ifxetex,ifluatex}
\usepackage{fixltx2e} % provides \textsubscript
\ifnum 0\ifxetex 1\fi\ifluatex 1\fi=0 % if pdftex
  \usepackage[T1]{fontenc}
  \usepackage[utf8]{inputenc}
\else % if luatex or xelatex
  \ifxetex
    \usepackage{mathspec}
  \else
    \usepackage{fontspec}
  \fi
  \defaultfontfeatures{Ligatures=TeX,Scale=MatchLowercase}
\fi
% use upquote if available, for straight quotes in verbatim environments
\IfFileExists{upquote.sty}{\usepackage{upquote}}{}
% use microtype if available
\IfFileExists{microtype.sty}{%
\usepackage{microtype}
\UseMicrotypeSet[protrusion]{basicmath} % disable protrusion for tt fonts
}{}
\usepackage[margin=1in]{geometry}
\usepackage{hyperref}
\hypersetup{unicode=true,
            pdftitle={Practical Machine Learning Course Project},
            pdfauthor={Roselyn Villacorte},
            pdfborder={0 0 0},
            breaklinks=true}
\urlstyle{same}  % don't use monospace font for urls
\usepackage{graphicx,grffile}
\makeatletter
\def\maxwidth{\ifdim\Gin@nat@width>\linewidth\linewidth\else\Gin@nat@width\fi}
\def\maxheight{\ifdim\Gin@nat@height>\textheight\textheight\else\Gin@nat@height\fi}
\makeatother
% Scale images if necessary, so that they will not overflow the page
% margins by default, and it is still possible to overwrite the defaults
% using explicit options in \includegraphics[width, height, ...]{}
\setkeys{Gin}{width=\maxwidth,height=\maxheight,keepaspectratio}
\IfFileExists{parskip.sty}{%
\usepackage{parskip}
}{% else
\setlength{\parindent}{0pt}
\setlength{\parskip}{6pt plus 2pt minus 1pt}
}
\setlength{\emergencystretch}{3em}  % prevent overfull lines
\providecommand{\tightlist}{%
  \setlength{\itemsep}{0pt}\setlength{\parskip}{0pt}}
\setcounter{secnumdepth}{0}
% Redefines (sub)paragraphs to behave more like sections
\ifx\paragraph\undefined\else
\let\oldparagraph\paragraph
\renewcommand{\paragraph}[1]{\oldparagraph{#1}\mbox{}}
\fi
\ifx\subparagraph\undefined\else
\let\oldsubparagraph\subparagraph
\renewcommand{\subparagraph}[1]{\oldsubparagraph{#1}\mbox{}}
\fi

%%% Use protect on footnotes to avoid problems with footnotes in titles
\let\rmarkdownfootnote\footnote%
\def\footnote{\protect\rmarkdownfootnote}

%%% Change title format to be more compact
\usepackage{titling}

% Create subtitle command for use in maketitle
\newcommand{\subtitle}[1]{
  \posttitle{
    \begin{center}\large#1\end{center}
    }
}

\setlength{\droptitle}{-2em}

  \title{Practical Machine Learning Course Project}
    \pretitle{\vspace{\droptitle}\centering\huge}
  \posttitle{\par}
    \author{Roselyn Villacorte}
    \preauthor{\centering\large\emph}
  \postauthor{\par}
      \predate{\centering\large\emph}
  \postdate{\par}
    \date{December 17, 2018}


\begin{document}
\maketitle

\subsection{Introduction}\label{introduction}

Using devices such as Jawbone Up, Nike FuelBanl, and Fitbit is now
possible to collect a large amount of data about personal activity
relatively inexpensively. These type of devices are part of the
quantified self mvement - a group of enthusiast who take measurements
about themselves regularly to improve their health, to find patterns in
their behaviour, or because they are tech geeks. One thing that people
regularly do is quantify how much of a particular activity they do, but
they rarely quantify how well they do it. This project goal is to use
data from accelerometers on the belt, forearm, arm, and dumbell of 6
participants. They were asked to perform barbell lifts correctly and
incorrectly in 5 different ways.

\subsubsection{The Data}\label{the-data}

The training data for this project are available at the following
website:
\url{https://d396qusza40orc.cloudfront.net/predmachlearn/pml-training.csv}

The test data are available here:
\url{https://d396qusza40orc.cloudfront.net/predmachlearn/pml-testing.csv}

The following data were classified in 5 different ways in lifting a
dumbell

\begin{enumerate}
\def\labelenumi{\arabic{enumi}.}
\tightlist
\item
  Class A - exactly according to the specification
\item
  Class B - throwing the elbows to the front
\item
  Class C - lifting the dumbell only half
\item
  Class D - lowering the dumbell only halfway
\item
  Class E - throwing the hips to the front
\end{enumerate}

\begin{verbatim}
## randomForest 4.6-14
\end{verbatim}

\begin{verbatim}
## Type rfNews() to see new features/changes/bug fixes.
\end{verbatim}

\begin{verbatim}
## 
## Attaching package: 'randomForest'
\end{verbatim}

\begin{verbatim}
## The following object is masked from 'package:ggplot2':
## 
##     margin
\end{verbatim}

\begin{verbatim}
## Rattle: A free graphical interface for data science with R.
## Version 5.2.0 Copyright (c) 2006-2018 Togaware Pty Ltd.
## Type 'rattle()' to shake, rattle, and roll your data.
\end{verbatim}

\begin{verbatim}
## 
## Attaching package: 'rattle'
\end{verbatim}

\begin{verbatim}
## The following object is masked from 'package:randomForest':
## 
##     importance
\end{verbatim}

\begin{verbatim}
## [1] 19622   160
\end{verbatim}

\begin{verbatim}
## [1]  20 160
\end{verbatim}

\begin{verbatim}
## [1] 19622    53
\end{verbatim}

\begin{verbatim}
## [1] 20 53
\end{verbatim}

\begin{verbatim}
##                matrix_tree_model.overall matrix_result.overall
## Accuracy                    7.041191e-01             0.9930669
## Kappa                       6.260949e-01             0.9912290
## AccuracyLower               6.911235e-01             0.9903250
## AccuracyUpper               7.168713e-01             0.9951940
## AccuracyNull                2.844617e-01             0.2844617
## AccuracyPValue              0.000000e+00             0.0000000
## McnemarPValue               2.540201e-53                   NaN
\end{verbatim}

\begin{verbatim}
##  1  2  3  4  5  6  7  8  9 10 11 12 13 14 15 16 17 18 19 20 
##  B  A  B  A  A  E  D  B  A  A  B  C  B  A  E  E  A  B  B  B 
## Levels: A B C D E
\end{verbatim}

\subsection{Including Plots}\label{including-plots}

You can also embed plots, for example:

decision\_tree\_model, echo=FALSE\}
fancyRpartPlot(decision\_tree\_model) ```

Note that the \texttt{echo\ =\ FALSE} parameter was added to the code
chunk to prevent printing of the R code that generated the plot.


\end{document}
